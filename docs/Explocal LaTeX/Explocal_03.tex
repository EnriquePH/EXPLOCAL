% !TeX encoding = ISO-8859-1
%-----------------------------------------------------------------------
% Filename: 	Explocal_EE.tex
% Description: 	PROYECTO DE UNA APLICACI�N INFORM�TICA 
%				PARA EL C�LCULO DE LAS PRINCIPALES 
%				CARACTER�STICAS TE�RICAS DE LOS EXPLOSIVOS
% Author:   	Enrique P�rez Herrero
% Date:     	13/Ene/2015
%-----------------------------------------------------------------------


\chapter{CONSIDERACIONES  PREVIAS}

Cuando  se  pretende  sustituir  por  otra,  o  crear  una  aplicaci�n  inform�tica, 
con  el  fin  de  satisfacer  una  necesidad  de  c�lculo,   se  incurre  en  dos  tipos  de  costes  diferentes:
Los  costes  de  desarrollo  y  los  costes  de  utilizaci�n. 
Aunque ambas  partidas  influyen  en  el  coste total  de  la  aplicaci�n, 
el  equipo  de  desarrollo  incurre  en  los  costes  de  desarrollo,  mientras  que  los  costes  de  utilizaci�n  son  cuesti�n  del  usuario.

El   presente  estudio  econ�mico  se  va  a  considerar  desde  el  punto  de  vista  del  equipo  de  desarrollo.

El  trabajo  del  equipo  de  desarrollo  influye  en  el  usuario  mediante  el  precio  de  venta  de  la  aplicaci�n  inform�tica.
El  precio  de  venta  debe  incluir  el  coste  de  desarrollo,  los  impuestos  
(IVA),  y  el  beneficio  de  la  empresa  de  desarrollo.     

El  estudio  econ�mico  que  se  va  realizar,  es  un  estudio  a  posteriori,  es  decir,  se  va  a  tener  en  cuenta  cu�nto  cuesta  la  versi�n  Beta  de  Explocal ,  ya  terminada.

Para   llevar  a  cabo  un  estudio  a  priori,  que  estime  el  coste  de  una   aplicaci�n  a  partir  de  una  breve  descripci�n  de  esta,  se debe  con  datos  econ�micos  del  desarrollo  de  aplicaciones  similares.    De  este  modo  es  posible  estimar  con  fiabilidad  el  tiempo  que  puede  llevar  finalizar,  por  ejemplo,  el  dise�o  y  la  codificaci�n.


\chapter{PRESUPUESTO  DE  DESARROLLO}

En  la  tabla \ref{TBEE01} , se  muestra  el  dinero  que  cuesta  pagar  al  equipo  de  desarrollo  de  software.  

El   tiempo  m�ximo  que  se  tarda  en  finalizar  la  versi�n  Beta  de  Explocal  es  de 1215 h  (Aproximadamente  7 meses  a  9 h/d�a).  En  realidad  algunas  de  las  tareas  se  pueden  realizar  de  forma  simult�nea,  con  lo  que  el  tiempo  de  ejecuci�n  del  proyecto  bajar�a  considerablemente.



\begin{table}[h]
\begin{tabular}{|l|c|r|r|}
\hline
\rowcolor[HTML]{FFCB2F} 
Personal:                & Sueldo: (PTA/h) & \multicolumn{1}{c|}{\cellcolor[HTML]{FFCB2F}Tiempo: (h)} & \multicolumn{1}{c|}{\cellcolor[HTML]{FFCB2F}Coste:  (PTA)} \\ \hline
T�cnico  en  explosivos. & 2.500           & 180                                                      & 450.000                                                    \\ \hline
Analista.                & 2.500           & 90                                                       & 225.000                                                    \\ \hline
Dise�ador                & 2.500           & 135                                                      & 337.500                                                    \\ \hline
Programadores.           & 2.000           & 510                                                      & 1.020.000                                                  \\ \hline
Mecan�grafo.             & 1.200           & 300                                                      & 360.000                                                    \\ \hline
\rowcolor[HTML]{FFFC9E} 
TOTAL PERSONAL:          & -               & 1.215                                                    & 2.392.500                                                  \\ \hline
\end{tabular}
\caption{Costes  de  personal}
\label{TBEE01}
\end{table}

\newpage

El  resto  de  costes  son  los  originados  por  la  necesidad  de  contar  con   el  equipo  f�sico  y  el  compilador  y  se  exponen  en  la  tabla \ref{TBEE02}.

\begin{table}[h]
\begin{tabular}{|l|c|r|}
\hline
\rowcolor[HTML]{FFCB2F} 
Material:          & Descripci�n:                              & \multicolumn{1}{l|}{\cellcolor[HTML]{FFCB2F}Coste: (PTA)} \\ \hline
Equipo  f�sico:    & Ordenador  486 DX2 66 MHz - 8 Mb RAM      & 169.400                                                   \\ \hline
                   & Impresora  HP Deskjet 520.                & 43.900                                                    \\ \hline
Programas:         & Compilador  C++  Borland  3.1             & 34.380                                                    \\ \hline
                   & Sistema  operativo Windows 3.1            & 6.900                                                     \\ \hline
                   & Procesador de  textos Microsoft Word  6.0 & 22.000                                                    \\ \hline
                   & Programa  ayudas  Hipertexto, HelpEdit    & 3.150                                                     \\ \hline
Material fungible. & 20  Disquetes $3�"$                      & 3.100                                                     \\ \hline
                   & 1  Cartucho de  tinta  impresora.         & 5.100                                                     \\ \hline
\rowcolor[HTML]{FFFC9E} 
TOTAL MATERIAL         & --                                        & 287.930                                                   \\ \hline
\end{tabular}
\caption{Costes  de  material}
\label{TBEE02}
\end{table}


Sumando  los  costes  de  material  con  los  de  personal,  obtendremos:
	
		\textbf{COSTE  TOTAL  DE  PROYECTO = 2.680.430 PTA}
