% !TeX encoding = ISO-8859-1
%-----------------------------------------------------------------------
% Filename: 	Explocal_EE.tex
% Description: 	PROYECTO DE UNA APLICACI�N INFORM�TICA 
%				PARA EL C�LCULO DE LAS PRINCIPALES 
%				CARACTER�STICAS TE�RICAS DE LOS EXPLOSIVOS
% Author:   	Enrique P�rez Herrero
% Date:     	13/Ene/2015
%-----------------------------------------------------------------------


\chapter{CONSIDERACIONES  PREVIAS}

Cuando  se  pretende  sustituir  por  otra,  o  crear  una  aplicaci�n  inform�tica, 
con  el  fin  de  satisfacer  una  necesidad  de  c�lculo,   se  incurre  en  dos  tipos  de  costes  diferentes:
Los  costes  de  desarrollo  y  los  costes  de  utilizaci�n. 
Aunque ambas  partidas  influyen  en  el  coste total  de  la  aplicaci�n, 
el  equipo  de  desarrollo  incurre  en  los  costes  de  desarrollo,  mientras  que  los  costes  de  utilizaci�n  son  cuesti�n  del  usuario.

El   presente  estudio  econ�mico  se  va  a  considerar  desde  el  punto  de  vista  del  equipo  de  desarrollo.

El  trabajo  del  equipo  de  desarrollo  influye  en  el  usuario  mediante  el  precio  de  venta  de  la  aplicaci�n  inform�tica.
El  precio  de  venta  debe  incluir  el  coste  de  desarrollo,  los  impuestos  
(IVA),  y  el  beneficio  de  la  empresa  de  desarrollo.     

El  estudio  econ�mico  que  se  va  realizar,  es  un  estudio  a  posteriori,  es  decir,  se  va  a  tener  en  cuenta  cu�nto  cuesta  la  versi�n  Beta  de  Explocal ,  ya  terminada.

Para   llevar  a  cabo  un  estudio  a  priori,  que  estime  el  coste  de  una   aplicaci�n  a  partir  de  una  breve  descripci�n  de  esta,  se debe  con  datos  econ�micos  del  desarrollo  de  aplicaciones  similares.    De  este  modo  es  posible  estimar  con  fiabilidad  el  tiempo  que  puede  llevar  finalizar,  por  ejemplo,  el  dise�o  y  la  codificaci�n.


\chapter{PRESUPUESTO  DE  DESARROLLO}
